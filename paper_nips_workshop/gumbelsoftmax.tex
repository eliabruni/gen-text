
\section{Gumbel-softmax distribution}

The softmax function can be used to parameterized a multinomial distribution
on a one-hot-encoding $d$-dimensional vector $\mathbf{y}$ in terms of a
continuous $d$-dimensional vector $\mathbf{h}$. Let $\mathbf{p}$ be a $d$-dimensional vector of probabilities
specifying the multinomial distribution on $\mathbf{y}$ with $p_i = p(y_i=1)$, $i=1,\ldots,d$. Then
\begin{align}
\mathbf{p} = \text{softmax}(\mathbf{h})\,\label{sec:gumbel:eq:0}
\end{align}
where $\text{softmax}(\cdot)$ returns here a $d$-dimensional vector with the output of the softmax function:
\begin{align}
\left[\text{softmax}(\mathbf{h})\right]_i = \frac{\exp(\mathbf{h}_i)}{\sum_{j=1}^K\exp(\mathbf{h}_j)}\,,\quad\text{for}\quad i = 1,\ldots,d\,. \label{sec:gumbel:eq:softmax_dim}
\end{align}
It can be shown that sampling $\mathbf{y}$ according to the previous multinomial distribution with probability vector 
given by (\ref{sec:gumbel:eq:0}) is the same as sampling $\mathbf{y}$ according to
\begin{align}
\mathbf{y} = \text{one\_hot}(\underset{i}{\arg\max} (h_i + g_i))\,,\label{sec:gumbel:eq:1}
\end{align}
where the $g_i$ are independent and follow a Gumbel distribution with zero location and unit scale.

The sample generated in (\ref{sec:gumbel:eq:1}) has gradient zero with respect to
$\mathbf{h}$ because the $\text{one\_hot}(\arg\max(\cdot))$
operator is not differentiable.
We propose to approximate this operator with a differentiable function based on the soft-max transformation \cite{jang2016categorical}.
In particular, we approximate $\mathbf{y}$ with 
\begin{align}
\mathbf{y} = \text{softmax}(1 / \tau (\mathbf{h} + \mathbf{g})))\,,\label{sec:gumbel:eq:2}
\end{align}
where $\tau$ is an inverse temperature parameter. When $\tau \rightarrow 0$, the samples generated by (\ref{sec:gumbel:eq:2})
have the same distribution as those generated by (\ref{sec:gumbel:eq:1}) and when $\tau \rightarrow \infty$,
the samples are always the uniform probability vector. For positive and finite values
of $\tau$ the samples generated by (\ref{sec:gumbel:eq:2}) are smooth and differentiable with respect to $\mathbf{h}$.

The probability distribution for (\ref{sec:gumbel:eq:2}), which is
parameterized by $\tau$ and $\mathbf{h}$, is called the Gumbel-softmax
distribution \cite{jang2016categorical}. A GAN on discrete data can then be
trained by using (\ref{sec:gumbel:eq:2}), starting with some relatively large
$\tau$ and then anealing it to zero during training.

