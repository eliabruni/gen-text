\documentclass{article}

% if you need to pass options to natbib, use, e.g.:
% \PassOptionsToPackage{numbers, compress}{natbib}
% before loading nips_2016
%
% to avoid loading the natbib package, add option nonatbib:
% \usepackage[nonatbib]{nips_2016}

%\usepackage{nips_2016}

% to compile a camera-ready version, add the [final] option, e.g.:
\usepackage[final]{nips_2016}

\usepackage[utf8]{inputenc} % allow utf-8 input
\usepackage[T1]{fontenc}    % use 8-bit T1 fonts
\usepackage{hyperref}       % hyperlinks
\usepackage{url}            % simple URL typesetting
\usepackage{graphicx} % more modern
\usepackage{booktabs}       % professional-quality tables
\usepackage{amsfonts}       % blackboard math symbols
\usepackage{nicefrac}       % compact symbols for 1/2, etc.
\usepackage{microtype}      % microtypography
\usepackage{floatrow}
\usepackage{amsmath}
\usepackage{amssymb}

\newcommand{\ica}{\hspace{0.15cm}}
\renewcommand{\arraystretch}{1.00}
\newfloatcommand{capbtabbox}{table}[][\FBwidth]

\title{GANS for Sequences of Discrete Elements \\with the Gumbel-softmax Distribution}

% The \author macro works with any number of authors. There are two
% commands used to separate the names and addresses of multiple
% authors: \And and \AND.
%
% Using \And between authors leaves it to LaTeX to determine where to
% break the lines. Using \AND forces a line break at that point. So,
% if LaTeX puts 3 of 4 authors names on the first line, and the last
% on the second line, try using \AND instead of \And before the third
% author name.

\author{
    Matt Kunser\\
    Alan Turing Institute
    \And
    Jos\'e Miguel Hern\'andez-Lobato\\
    University of Cambridge
  %% examples of more authors
  %% \And
  %% Coauthor \\
  %% Affiliation \\
  %% Address \\
  %% \texttt{email} \\
  %% \AND
  %% Coauthor \\
  %% Affiliation \\
  %% Address \\
  %% \texttt{email} \\
  %% \And
  %% Coauthor \\
  %% Affiliation \\
  %% Address \\
  %% \texttt{email} \\
  %% \And
  %% Coauthor \\
  %% Affiliation \\
  %% Address \\
  %% \texttt{email} \\
}

\begin{document}
% \nipsfinalcopy is no longer used

\maketitle

\begin{abstract}
Generative Adversarial Networks (GAN) have liminations when the goal is to
generate sequences of discrete elements. The reason for this is that
samples from a distribution on discrete objects such as the multinomial are
not differentiable with respect to the distribution parameters. This problem
can be avoided by using the Gumbel-softmax distribution, which is a continuous
approximation to a multinomial distribution parameterized in terms of the
softmax function. In this work, we evaluate the performance of GANs based on
recurrent neural networks with Gumbel-softmax output distributions in the task
of generating sequences of discrete elements.
\end{abstract}

\section{Introduction}

Generative adversarial networks (GANs) are methods for generating synthetic
data with similar statistical properties as the real one
\cite{goodfellow2014generative}. In the GAN methodology a discriminative neural network D is
trained to distinguish whether a given data instance is synthetic or real,
while a generative network G is jointly trained to confuse D by generating high
quality data. This approach has been very sucessful in computer vision tasks for
generating samples of natural images
\cite{denton2015deep,dosovitskiy2016generating,radford2016}.

GANs work by propagating gradients back from the discriminator D through the
generated samples to the generator G. This is perfectly feasible when the
generated data is continuous such as in the examples with images mentioned
above. However, a lot of data exists in the form of squences of discrete items.
For example, text sentences \cite{Bowman2016}, molecules encoded in the SMILE language \cite{gomez2016automatic}, etc. In these
cases, the discrete data is not differentiable and the backpropagated gradients
are always zero. 

Discrete data, encoded using a one-hot representation, can be sampled from a
multinomial distribution with probabilities given by the output of a softmax
function. The resulting sampling process is not differentiable.  However, we can obtain
a differentiable approximation by sampling from the Gumbel-softmax distribution
\cite{jang2016categorical}. This distribution has been previously used to train
variatoinal autoencoders with discrete latent variables \cite{jang2016categorical}. Here, we propose to
use it to train GANs on sequences of discrete tokens and we evaluate its
performance in this setting.



\section{Gumbel-softmax distribution}

The softmax function can be used to parameterized a multinomial distribution
on a one-hot-encoding $d$-dimensional vector $\mathbf{y}$ in terms of a
continuous $d$-dimensional vector $\mathbf{x}$. Let $\mathbf{p}$ be a $d$-dimensional vector of probabilities
specifying the multinomial distribution on $\mathbf{y}$ with $p_i = p(y_i=1)$, $i=1,\ldots,d$. Then
\begin{align}
\mathbf{p} = \text{softmax}(\mathbf{x})\,\label{sec:gumbel:eq:0}
\end{align}
where $\text{softmax}(\cdot)$ returns here a $d$-dimensional vector with the output of the softmax function:
\begin{align}
\left[\text{softmax}(\mathbf{x})\right]_i = \frac{\exp(\mathbf{x}_i)}{\sum_{j=1}^K\exp(\mathbf{x}_j)}\,,\quad\text{for}\quad i = 1,\ldots,d\,.
\end{align}
It can be shown that sampling $\mathbf{y}$ according to the previous multinomial distribution with probability vector 
given by (\ref{sec:gumbel:eq:0}) is the same as sampling $\mathbf{y}$ according to
\begin{align}
\mathbf{y} = \text{one\_hot}(\underset{i}{\arg\max} (x_i + z_i))\,,\label{sec:gumbel:eq:1}
\end{align}
where the $z_i$ are independent and follow a Gumbel distribution with zero location and unit scale.

The sample generated in (\ref{sec:gumbel:eq:1}) has gradient zero with respect to
$\mathbf{x}$ because the $\text{one\_hot}(\arg\max(\cdot))$
operator is not differentiable.
We propose to approximate this operator with a differentiable function based on the soft-max transformation \cite{jang2016categorical}.
In particular, we approximate $\mathbf{y}$ with 
\begin{align}
\mathbf{y} = \text{softmax}(1 / \tau (\mathbf{x} + \mathbf{z})))\,,\label{sec:gumbel:eq:2}
\end{align}
where $\tau$ is an inverse temperature parameter. When $\tau \rightarrow 0$, the samples generated by (\ref{sec:gumbel:eq:2})
have the same distribution as those generated by (\ref{sec:gumbel:eq:1}) and when $\tau \rightarrow \infty$,
the samples are always the uniform probability vector. For positive and finite values
of $\tau$ the samples generated by (\ref{sec:gumbel:eq:2}) are smooth and differentiable with respect to $\mathbf{x}$.

The probability distribution for (\ref{sec:gumbel:eq:2}), which is
parameterized by $\tau$ and $\mathbf{x}$, is called the Gumbel-softmax
distribution \cite{jang2016categorical}. A GAN on discrete data can then be
trained by using (\ref{sec:gumbel:eq:2}), starting with some relatively large
$\tau$ and then anealing it to zero during training.



\section{A recurrent neural network for discrete sequences}

\section{Experiments}


\bibliography{references}
\bibliographystyle{plain}

\end{document}
